%zu "untersuchen". Mir geht es persönlich darum, wie dieses Produkt
%im realen Alltag einzusetzen ist und wie es technisch funktioniert
%(whitepaper lesen/überfliegen!).

%Dafür hätte ich gerne ein Tutorial, das genau diese Punkte als
%"Tutorial"
%behandelt, sodass ich dieses im Unterricht (unter Linux) einsetzen
%könnte.
%Speziell interessant ist der Einsatz in einem Container.
\chapter{WireGuard}
\begin{figure}[htbp]
  \centering
  \includesvg[scale=0.20]{images/wireguard.svg}
  \caption{svg image}
\end{figure}

\section{Einführung} % Was kann ich damit machen?
WireGuard ist ein extrem einfaches und dennoch schnelles und modernes VPN-Protkoll, welches eine sichere Lösung für das VPN-Tunneling bieten soll. Es ist darauf ausgelegt, leistungsfähiger, einfacher und nützlicher als die Konkurrenz z.B. IPsec, OpenVPN  zu sein. WireGuard ist als Allzweck-VPN konzipiert, das sowohl auf eingebetteten Schnittstellen als auch auf Supercomputern ausgeführt werden kann und für viele verschiedene Umstände geeignet ist.  \newline\newline
Ursprünglich wurde WireGuard für den Linux-Kernel veröffentlicht, ist jedoch nun plattformübergreifend (Windows, MacOS, BSD, iOS, Android) weithin einsetzbar. Derzeit wird WireGuard stark weiterentwickelt, aber kann jetzt schon als die sicherste, benutzerfreundlichste und einfachste VPN-Lösung in der Branche angesehen werden.

\section{Installation} % Einsatz im Realen Alltag

\section{Verwendung} % Einsatz im Realen Alltag

\section{Technische Funktionalität}

