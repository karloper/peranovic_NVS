\documentclass[german,oneside,color]{htldipl}

\graphicspath{{images/}}
\usepackage[paper=a4paper,margin=3cm]{geometry}

\makeindex[title=Index]
\makeindex[name=allgemein, title=Allgemeiner Index]
\makeindex[name=name,title={Autoren Index}]
\makeindex[name=title,columns=1,title={Literatur Index}]
\indexsetup{level=\subsection*, toclevel=subsection, noclearpage}


\makeatletter
\@ifpackageloaded{biblatex_legacy}
  {\DeclareIndexNameFormat{default}{%
     \usebibmacro{index:name}{\index[name]}{#1}{#3}{#5}{#7}}}
  {\DeclareIndexNameFormat{default}{%
     \usebibmacro{index:name}{\index[name]}
       {\namepartfamily}
       {\namepartgiven}
       {\namepartprefix}
       {\namepartsuffix}}}
\makeatother

\DeclareIndexFieldFormat{indextitle}{%
  \usebibmacro{index:title}{\index[title]}{#1}}

\renewbibmacro*{bibindex}{%
  \ifbibindex
    {\indexnames{author}%
     \indexnames{editor}%
     \indexnames{translator}%
     \indexnames{commentator}%
     \indexfield{indextitle}}
    {}}

\makeatletter
\DeclareCiteCommand{\repeatfootcite}[\cbx@wrap]
  {\gdef\cbx@keys{}}
  {\xappto\cbx@keys{\thefield{entrykey},}}
  {}
  {\ifcsundef{cbx@lastin@\cbx@keys @\strfield{postnote}}
     {\csnumgdef{cbx@lastin@\cbx@keys @\strfield{postnote}}{-1}}{}%
   \ifsamepage{\value{instcount}}{\csuse{cbx@lastin@\cbx@keys @\strfield{postnote}}}
     {\footnotemark[\csuse{cbx@lastfn@\cbx@keys @\strfield{postnote}}]}
     {\xappto\cbx@cite{\noexpand\footcite%
        [\thefield{prenote}][\thefield{postnote}]{\cbx@keys}%
        \csnumgdef{cbx@lastfn@\cbx@keys @\strfield{postnote}}{\value{\@mpfn}}%
        \csnumgdef{cbx@lastin@\cbx@keys @\strfield{postnote}}{\value{instcount}}}}}

\newrobustcmd{\cbx@wrap}[1]{#1\cbx@cite\gdef\cbx@cite{}}
\def\cbx@cite{}
\makeatother
\makeglossaries
\loadglsentries{glossary}
\addbibresource{literatur.bib} 

\begin{document}
\abteilung{Informatik}
\schwerpunkt{} % eintragen, wenn vorhanden
\studienort{Wiener Neustadt} \schule{HTBLuVA Wiener Neustadt}
\schullogo{htl.jpeg} \abgabejahr{2019/20}
\betreuerB{} \betreuerC{} \betreuerD{}
\schuelerC{} \evidenzC{} \subthemaC{}
\schuelerD{} \evidenzD{} \subthemaD{}
\schuelerE{} \evidenzE{} \subthemaE{}

%                                 wichtig
\title{WireGuard}
\schuelerA{Karlo PERANOVIC}
\evidenzA{5BHIF-18}
\subthemaA{WireGuard für Unterrichtseinsatz aufbereiten}
\schuelerB{}
\evidenzB{}
\subthemaB{}
\betreuerA{Dipl.-Ing. Dr. Günter Kolousek}

\frontmatter
\maketitle
\tableofcontents

% Hauptteil
\mainmatter     
\chapter{Einleitung}



% Anhang
\appendix
\chapter{Anhang}
	

% Automatisch:: Glossar, Index, Literaturverzeichnis; nichts ändern
\clearpage
\printglossaries
\clearpage
\chapter*{Index}
\addcontentsline{toc}{chapter}{Index}
\printindex[allgemein]
\printindex
\printindex[name]
\printindex[title]
\clearpage
\addcontentsline{toc}{chapter}{\bibname}
\printbibliography
\end{document}
